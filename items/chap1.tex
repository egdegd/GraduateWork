\section{Обзорный раздел}

\subsection{Мотивация}
На практике популярным подходом для решения трудных комбинаторных задач является кодирование этой задачи в~КНФ и запуск солвера для задачи выполнимости (SAT-солвер). 
Есть две основные причины, почему такой подход хорошо справляется для многих сложных задач: 
современные SAT-солверы  чрезвычайно эффективны и 
многие комбинаторные задачи естественно записываются в~КНФ. В тоже время, КНФ-кодировка не уникальна, и, обычно, ее~выбирают эмпирически. 
Кроме того, не существует такого понятия, как наилучшая кодировка, так как это также зависит от выбранного SAT-солвера. Прествич~\cite{DBLP:series/faia/Prestwich09} делает обзор различных способов перевода задач в КНФ и обсуждает их желательные свойства, как с теоретической точки зрения, так и с практической. \\
Уже для такой простой функции, как четность $x_1 \oplus x_2 \oplus \dotsb \oplus x_n$, 
в действительности, не ясно, как кодировать ее в~КНФ (чтобы SAT-солверу было проще работать). 
Функция четности часто возникает в криптографии (хэш - функция, потоковые шифры, и т.д.). 
Известно, что минимальное число клозов в~КНФ, вычисляющей функцию четности - это $2^{n-1}$. 
С ростом~$n$ это число становится неприменимо на практике. 
Стандартный способ уменьшить размер кодировки - это использовать дополнительные переменные. 
Давайте введем $s$ дополнительны переменных $y_1, \dotsc, y_s$, и разобьем множество входных переменных на~$s+1$ блок размера не больше $\lceil n/(s+1) \rceil$: $\{x_1, x_2, \dotsc, x_n\}=X_1 \sqcup X_2 \sqcup \dotsb \sqcup X_{s+1}$, 
тогда функцию четности можно закодировать следующим способом: 
\begin{multline}\label{eq:blocks}
	\left(y_1=\bigoplus_{x \in X_1}x\right),
	\left(y_2=y_1 \oplus \bigoplus_{x \in X_2}x\right), \dotsc,\\
	\left(y_s=y_{s-1} \oplus \bigoplus_{x \in X_s}x\right),
	\left(1=y_s \oplus \bigoplus_{x \in X_{s+1}}x\right).
\end{multline}
Значение для параметра $s$ обычно определяется экспериментально. Например, Прествич~~\cite{DBLP:series/faia/Prestwich09} пишет, что значение $s = 10$ дает наилучшие результаты при решении задачи "The Minimal Disagreement Parity Problem"\ \cite{Crawford1995TheMD}, используя SAT-солвер, основанный на локальном поиске.

Простая конструкция, приведенная выше, влечет несколько верхних оценок 
на число клозов $m$, количество дополнительных переменных $s$ и ширину клозов $k$:
\begin{description}
	\item[Ограниченное количество дополнительных переменных:]
	Используя $s$~дополнительных переменных можно закодировать функцию четности либо как 
	КНФ с не более чем \[m \le (s+1)2^{\lceil n/(s+1) \rceil+2-1} \le 4(s+1)2^{n/(s+1)}\] клозами, либо как $k$-КНФ, где  \[k=2+{\lceil n/(s+1) \rceil} \le 3+n/(s+1) \, .\]
	\item[Неограниченное количество дополнительных переменных:] можно закодировать функцию четности в виде КНФ, использую не более чем $4n$~клозов (для этого надо взять $s=n-1$ дополнительных переменных; тогда каждую из~$n$~функций в ~\eqref{eq:blocks} можно записать в КНФ, используя не больше $4$ клозов).
\end{description}
\subsection{Результаты}
В этой работе показывается, что верхние оценки, упомянутые выше, являются практически оптимальными. 
\begin{theorem}\label{thm:main}
	Пусть $F$ - КНФ, кодирующая функцию $PAR_n$ с помощью $m$ клозов, 
	$s$ дополнительных переменных, и максимальной длиной клоза равной $k$.
	\begin{enumerate}
		\item Параметры $s$~и~$m$ не могут быть слишком маленькие одновременно:
		\begin{equation}\label{eq:sm}
			m \ge \Omega\left(\frac{s+1}{n} \cdot 2^{n/(s+1)}\right) \, .
		\end{equation}
		\item Параметры $s$~и~$k$ не могут быть слишком маленькие одновременно:
		\begin{equation}\label{eq:sw}
			k \ge n/(s+1) \, .
		\end{equation}
		\item Параметр $m$ не может быть слишком маленький:
		\begin{equation}\label{eq:m}
			m \ge 3n-9 \, .
		\end{equation}
	\end{enumerate}
\end{theorem}

\subsection{Методы}
Нижняя оценка $m \ge \Omega((s+1)2^{n/(s+1)}/n)$ получена 
из~Satisfiability Coding Lemma авторов Патури, Пудлак и Зейн~\cite{DBLP:journals/cjtcs/PaturiPZ99}. 
Эта лемма позволяет доказать $2^{\sqrt{n}}$ нижнюю оценку на размер схемы глубины $3$, вычисляющую функцию четности. 
Отметим, что нижняя оценка $m \ge \Omega((s+1)2^{n/(s+1)}/n)$ влечет нижнюю $2^{\Omega(\sqrt n)}$ практически сразу,
при этом в обратную сторону такое следствие не ясно.

Для доказательства нижней оценки $m \ge 3n - 9$, в этой работе была тщательно проанализирована структура КНФ кодировки.

\subsection{Ранние работы}
Голдсмит, Леви и Манденк~\cite{DBLP:journals/sigact/GoldsmithLM96} показали много результатов для различных вычислительных моделей с дополнительными переменными. 
Обзор известных подходов для КНФ кодировок представлены у Прествича~\cite{DBLP:series/faia/Prestwich09}.
Два недавних результата, близких к результатам этой работы следующие.
Моридзуми~\cite{DBLP:conf/cocoon/Morizumi15} доказал, что дополнительны входные переменные не помогают в модели булевых схем над базисом $U_2$ (множество всех бинарных функций, кроме эквивалентности и исключающего или) для вычисления функции четности: 
с и без дополнительных переменных минимальный размер схемы, вычисляющей функцию четности, равен $3(n - 1)$.
Киосера, Савицкий, Ворель~\cite{DBLP:journals/tcs/KuceraSV19} продемонстрировали практически совпадающие нижнюю и верхнюю границу на размер КНФ кодировки булевой функции at-most-one ($[x_1+\dotsb+x_n \le 1]$). Синз~\cite{DBLP:conf/cp/Sinz05} получил линейную нижнюю оценку на КНФ кодировку булевой функции at-most-k.
