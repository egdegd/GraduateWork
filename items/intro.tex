\specialsection{Введение}

Многие годы ученые в области компьютерных наук придумывают алгоритмы для ускорения решения различных задач. В теории сложности существует иерархия классов, основными представителями которой являются классы $P$ и $NP$. Класс $P$ - класс языков (задач), разрешимых на детерминированной машине Тьюринга за полиномиальное время. Класс $NP$ - класс языков (задач), ответ на который можно проверить за полиномиальное время. Вопрос о равенстве классов $P$ и $NP$ - это одна из центральных открытых проблем теории алгоритмов \cite{?} \todo{Добавить}. Каноническим представителем класса $NP$ является задача выполнимости булевой формулы в конъюктивно-нормальной форме (КНФ). Несмотря на теоретическую сложность этой задачи, современные солверы работают очень быстро. К задаче выполнимости на практике сводятся огромное количество других задач. При этом сведении размер задачи может экспоненциально увеличиться. В связи с этим, возникает вопрос минимизации количества клозов при кодировки функций в виде КНФ.