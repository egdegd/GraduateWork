\section{Нижние оценки на КНФ-кодировки функции четности}
В этой главе будут приведено доказательство Теоремы~\ref{thm:main}.
Главное свойство функции четности, которым мы будем пользоваться - это высокая чувствительность 
(каждое выполняющее означивание изолированно):
для всех $i \in [n]$ и всех $x,x' \in \{0,1\}^n$, которые отличаются только в $i$-ой позиции,
$\PAR(x) \neq \PAR(x')$.
Это обозначает, что если КНФ $F$ вычисляет $\PAR$ и $F(x) = 1$, то $F$ обязана содержать клоз, 
который выполняется только переменной $x_i$.
Как и в~\cite{DBLP:journals/cjtcs/PaturiPZ99}, назовем такой клоз критическим по отношению к $(x,i)$.
Эта обозначение естественно расширяется для КНФ-кодировки. Пусть $F(x, y)$ - это КНФ-кодировка функции четности.
Тогда для всех $(x, y)$ таких, что $F(x, y) = 1$, и всех $i \in [n]$ верно следующее: $F$ содержит клоз,
который станет невыполненным, если поменять значение бита $x_i$. 
Будем называть его критическим клозом по отношению к $(x, y, i)$.
\subsection{Ограниченное количество дополнительных переменных}
Для доказательства нижней оценки $m \ge \Omega((s+1)2^{n/(s+1)}/n)$, адаптируем доказательство нижней оценки $\Omega(n^{1/4}2^{\sqrt n})$ на схему глубины $3$, вычисляющую $\PAR_n$ 
от Патури, Пудлак и Зейн~\cite{DBLP:journals/cjtcs/PaturiPZ99}. 
Рассмотрим КНФ $F(x_1, \dotsc, x_n)$.
Для каждого изолированного выполняющего означивания~$x \in \{0,1\}^n$ функции $F$ и каждого~$i \in [n]$, зафиксируем самые короткий клоз по отношению к $(x,i)$ и обозначим его $C_{F,x,i}$.
Теперь для изолированного выполняющего означивания $x$, определим его вес по отношению к ~$F$, как 
\[w_F(x) = \sum\limits_{i=1}^n \frac{1}{|C_{F,x,i}|} \, .\]

\begin{lemma}[Лемма~5 из~\cite{DBLP:journals/cjtcs/PaturiPZ99}]\label{lemma:isolatedweight}
	Для всех~$\mu$ у $F$~может быть не более $2^{n - \mu}$ изолированных выполняющих означиваний 
	веса хотя бы $\mu$.
\end{lemma}

\begin{proof}[Доказательство~\eqref{eq:sm}, $m \ge \Omega\left(\frac{s+1}{n} \cdot 2^{n/(s+1)}\right)$]
	
\end{proof}
\subsection{Ширина дизъюнктов}
\subsection{Неограниченное количество дополнительных переменных}